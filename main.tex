%% Run LaTeX on this file several times to get Table of Contents,
%% cross-references, and citations.

\documentclass[11pt]{book}
\usepackage{gvv-book}
%\usepackage{Wiley-AuthoringTemplate}
\usepackage[sectionbib,authoryear]{natbib}% for name-date citation comment the below line
%\usepackage[sectionbib,numbers]{natbib}% for numbered citation comment the above line

%%********************************************************************%%
%%       How many levels of section head would you like numbered?     %%
%% 0= no section numbers, 1= section, 2= subsection, 3= subsubsection %%
\setcounter{secnumdepth}{3}
%%********************************************************************%%
%%**********************************************************************%%
%%     How many levels of section head would you like to appear in the  %%
%%				Table of Contents?			%%
%% 0= chapter, 1= section, 2= subsection, 3= subsubsection titles.	%%
\setcounter{tocdepth}{2}
%%**********************************************************************%%

%\includeonly{ch01}
\makeindex

\begin{document}

\frontmatter
%%%%%%%%%%%%%%%%%%%%%%%%%%%%%%%%%%%%%%%%%%%%%%%%%%%%%%%%%%%%%%%%
%% Title Pages
%% Wiley will provide title and copyright page, but you can make
%% your own titlepages if you'd like anyway
%% Setting up title pages, type in the appropriate names here:

\booktitle{The Navic Standard}

\subtitle{Through Python}

\AuAff{G. V. V. Sharma}


%% \\ will start a new line.
%% You may add \affil{} for affiliation, ie,
%\authors{Robert M. Groves\\
%\affil{Universitat de les Illes Balears}
%Floyd J. Fowler, Jr.\\
%\affil{University of New Mexico}
%}

%% Print Half Title and Title Page:
%\halftitlepage
\titlepage

%%%%%%%%%%%%%%%%%%%%%%%%%%%%%%%%%%%%%%%%%%%%%%%%%%%%%%%%%%%%%%%%
%% Copyright Page

\begin{copyrightpage}{2023}
%Title, etc
\end{copyrightpage}

% Note, you must use \ to start indented lines, ie,
% 
% \begin{copyrightpage}{2004}
% Survey Methodology / Robert M. Groves . . . [et al.].
% \       p. cm.---(Wiley series in survey methodology)
% \    ``Wiley-Interscience."
% \    Includes bibliographical references and index.
% \    ISBN 0-471-48348-6 (pbk.)
% \    1. Surveys---Methodology.  2. Social 
% \  sciences---Research---Statistical methods.  I. Groves, Robert M.  II. %
% Series.\\

% HA31.2.S873 2004
% 001.4'33---dc22                                             2004044064
% \end{copyrightpage}

%%%%%%%%%%%%%%%%%%%%%%%%%%%%%%%%%%%%%%%%%%%%%%%%%%%%%%%%%%%%%%%%
%% Only Dedication (optional) 

%\dedication{To my parents}

\tableofcontents

%\listoffigures %optional
%\listoftables  %optional

%% or Contributor Page for edited books
%% before \tableofcontents

%%%%%%%%%%%%%%%%%%%%%%%%%%%%%%%%%%%%%%%%%%%%%%%%%%%%%%%%%%%%%%%%
%  Contributors Page for Edited Book
%%%%%%%%%%%%%%%%%%%%%%%%%%%%%%%%%%%%%%%%%%%%%%%%%%%%%%%%%%%%%%%%

% If your book has chapters written by different authors,
% you'll need a Contributors page.

% Use \begin{contributors}...\end{contributors} and
% then enter each author with the \name{} command, followed
% by the affiliation information.

% \begin{contributors}
% \name{Masayki Abe,} Fujitsu Laboratories Ltd., Fujitsu Limited, Atsugi, Japan
%
% \name{L. A. Akers,} Center for Solid State Electronics Research, Arizona State University, Tempe, Arizona
%
% \name{G. H. Bernstein,} Department of Electrical and Computer Engineering, University of Notre Dame, Notre Dame, South Bend, Indiana; formerly of
% Center for Solid State Electronics Research, Arizona
% State University, Tempe, Arizona 
% \end{contributors}

%%%%%%%%%%%%%%%%%%%%%%%%%%%%%%%%%%%%%%%%%%%%%%%%%%%%%%%%%%%%%%%%
% Optional Foreword:

%\begin{foreword}
%\lipsum[1-2]
%\end{foreword}

%%%%%%%%%%%%%%%%%%%%%%%%%%%%%%%%%%%%%%%%%%%%%%%%%%%%%%%%%%%%%%%%
% Optional Preface:

%\begin{preface}
%\lipsum[1-1]
%\prefaceauthor{}
%\where{place\\
% date}
%\end{preface}

% ie,
% \begin{preface}
% This is an example preface.
% \prefaceauthor{R. K. Watts}
% \where{Durham, North Carolina\\
% September, 2004}

%%%%%%%%%%%%%%%%%%%%%%%%%%%%%%%%%%%%%%%%%%%%%%%%%%%%%%%%%%%%%%%%
% Optional Acknowledgments:

%\acknowledgments
%\lipsum[1-2]
%\authorinitials{I. R. S.}  

%%%%%%%%%%%%%%%%%%%%%%%%%%%%%%%%
%% Glossary Type of Environment:

% \begin{glossary}
% \term{<term>}{<description>}
% \end{glossary}

%%%%%%%%%%%%%%%%%%%%%%%%%%%%%%%%
%\begin{acronyms}
%\acro{ASTA}{Arrivals See Time Averages}
%\acro{BHCA}{Busy Hour Call Attempts}
%\acro{BR}{Bandwidth Reservation}
%\acro{b.u.}{bandwidth unit(s)}
%\acro{CAC}{Call / Connection Admission Control}
%\acro{CBP}{Call Blocking Probability(-ies)}
%\acro{CCS}{Centum Call Seconds}
%\acro{CDTM}{Connection Dependent Threshold Model}
%\acro{CS}{Complete Sharing}
%\acro{DiffServ}{Differentiated Services}
%\acro{EMLM}{Erlang Multirate Loss Model}
%\acro{erl}{The Erlang unit of traffic-load}
%\acro{FIFO}{First in - First out}
%\acro{GB}{Global balance}
%\acro{GoS}{Grade of Service}
%\acro{ICT}{Information and Communication Technology}
%\acro{IntServ}{Integrated Services}
%\acro{IP}{Internet Protocol}
%\acro{ITU-T}{International Telecommunication Unit -- Standardization sector}
%\acro{LB}{Local balance}
%\acro{LHS}{Left hand side}
%\acro{LIFO}{Last in - First out}
%\acro{MMPP}{Markov Modulated Poisson Process}
%\acro{MPLS}{Multiple Protocol Labeling Switching}
%\acro{MRM}{Multi-Retry Model}
%\acro{MTM}{Multi-Threshold Model}
%\acro{PASTA}{Poisson Arrivals See Time Averages}
%\acro{PDF}{Probability Distribution Function}
%\acro{pdf}{probability density function}
%\acro{PFS}{Product Form Solution}
%\acro{QoS}{Quality of Service}
%\acro{r.v.}{random variable(s)}
%\acro{RED}{random early detection}
%\acro{RHS}{Right hand side}
%\acro{RLA}{Reduced Load Approximation}
%\acro{SIRO}{service in random order}
%\acro{SRM}{Single-Retry Model}
%\acro{STM}{Single-Threshold Model}
%\acro{TCP}{Transport Control Protocol}
%\acro{TH}{Threshold(s)}
%\acro{UDP}{User Datagram Protocol}
%\end{acronyms}

\setcounter{page}{1}

\begin{introduction}
This book introduces the NAVIC communication standard through Python exercises

\end{introduction}

\mainmatter
\chapter{Design Parameters}
\section{The Frequency Bands}

\chapter{Channel Modelling}
The phenomena modelled in the satellite communication channel are Doppler shift, delay, power scaling and thermal noise at the receiver.
\section{Doppler shift}
Due to relative motion between the satellites and the receiver, the transmitted signals undergo a frequency shift before arriving at the receiver. This shift %
in frequency is called Doppler shift and can be computed as
\begin{equation}
    f_{shift} = f_{d}-f_{c} = \brak{\frac{V_{rel}}{c-V_{S,dir}}}f_{c}  
\end{equation}
where,

$f_{Shift}$ = Frequency shift due to Doppler

$f_{d}$ = Frequency observed at receiver

$f_{c}$ = Carrier frequency at transmitter

$V_{rel}$ = Relative velocity of transmitter and receiver

$V_{S,dir}$ = Velocity of satellite along radial direction

$c$ = Speed of light

$V_{rel}$ is given by
\begin{align}
    V_{rel} &= V_{S,dir} - V_{R,dir}
\end{align}
where,

$V_{R,dir}$ = Velocity of receiver along radial direction

$V_{R,dir}$ and $V_{S,dir}$ are given by
\begin{align}
    V_{R,dir} &= \vec{V}_{R} \cdot \hat{\vec{dir}}\\
    V_{D,dir} &= \vec{V}_{S} \cdot \hat{\vec{dir}}
\end{align}
where,

$\hat{\vec{dir}}$ = Unit vector from satellite to receiver i.e. radial direction

$\vec{V_{S}}$ = Velocity of satellite

$\vec{V_{R}}$ = Velocity of receiver

$\hat{\vec{dir}}$ is given by
\begin{align}
    \hat{\vec{dir}} = \frac{\vec{P_{S}}-\vec{P_{R}}}{\norm{\vec{P_{S}}-\vec{P_{R}}}}
\end{align}
where,

$\vec{P_{S}}$ = Position of satellite

$\vec{P_{R}}$ = Position of receiver


The Doppler shift is introduced by muliplying the satellite signal with a complex exponential,
\begin{equation}
    x_{Shift}\sbrak{n} = x\sbrak{n}e^{-2 \pi j \brak{f_{c}+f_{Shift}} n t_{s}}
\end{equation}
where,

$x_{Shift}\sbrak{n}$ = Doppler shifted signal

$x\sbrak{n}$ = Satellite signal

$t_{s}$ = Sampling period

\section{Delay}
Since there is a finite distance between the satellite and the receiver, the signal at the reciever is a delayed version of the transmitted signal. This delay is given by
\begin{equation}
    D_{s} = \frac{d}{c}f_{s} 
\end{equation}
where,

$D_{s}$ = Total delay in samples

$d$ = Distance between satellite and receiver

$c$ = Speed of light

$f_{s}$ = Sampling rate

The total delay on the satellite signal is modeled in two steps. First, a static delay is modeled which does not change with time and it is always an integer number of samples. Then, %
a variable delay is modeled which can be a rational number of samples. While modelling the static delay, the entire delay is not introduced so that variable delay modelling handles the remaining %
delay.

To introduce the static delay, the samples are read from a queue whose size is the desired static delay length. When samples are read from the queue, an equal number of new samples are %
updated in the queue. To introduce the variable delay, the signal is passed through an all-pass FIR filter with an almost constant phase response. Its coefficients are calculated %
using the delay value required.

\section{Power Scaling}
When a transmitting antenna transmits radio waves to a receiving antenna, the radio wave power received is given by,
\begin{equation}
    P_r = P_t D_t D_r \brak{\frac{1}{4 \pi \brak{f_c + f_{Shift}} D}}^2
\end{equation}
where,

$P_r$ = Received power

$P_t$ = Transmitted power

$D_t$ = Directivity of transmitting antenna 

$D_r$ = Directivity of receiving antenna 

$D$ = Total delay in seconds

To scale the received signal as per the received power calculated,
\begin{equation}
    x_{Scaled}\sbrak{n} = \frac{\sqrt{P_r}}{\operatorname{rms}\brak{x\sbrak{n}}}x\sbrak{n}
\end{equation}   

\section{Thermal noise}
The thermal noise power at the receiver is given by,
\begin{equation}
    N_r = k T B
\end{equation}
where,

$N_r$ = Noise power in watts

$k$ = Boltzmann's constant

$T$ = Temperature in Kelvin

$B$ = Bandwidth in Hz

AWGN (Additive White Guassian Noise) samples with zero mean and variance $N_r$ are generated and added to the satellite signal to model thermal noise at receiver.

The functions necessary to model the channel are present in the below code,
\begin{lstlisting}
    codes/channelmodel/channelmodel.py
\end{lstlisting}


\end{document}

 
