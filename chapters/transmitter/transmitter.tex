\section{Frame structure}
NavIC master frame consists of 2400 symbols, divided to four subframes. Each subframe is 600 symbols long. Subframes 1 and 2 transmit fixed navigation parameters. Subframe 3 and 4 transmit secondary navigation parameters in the form of messages. Each subframe is 292 bits long without FEC encoding and sync word. It starts with TLM word of 8 bits. Ends with 24 bit Cyclic Redundancy Check(CRC) followed by 6 tail bits. In subframes 1 and 2 navigation data is alloted 232 bits, starting from bit 31. In subframe 3 and 4, 220 bits are alloted starting from bit 37. For detailed structure of subframes, refer to chapter 5.9 in the doc
\subsection{Cyclic Redundancy Check(CRC)}
The parity coding of data signal follows 24Q polynomial for each subframe. 24 bits of CRC parity will provide protection against burst as well as random errors with undetected eroor probability of $2^{-24}$ for all channel bit error probabilities 0.5
\begin{equation}
    g(X) = \sum_{i = 0}^{24}g_{i}X^i\;\;
    g_{i}=1\; for\; i = 0,1,3,4,5,6,7,10,11,14,17,18,23,24
\end{equation}
\section{Encoding}
The navigation data subframe of 292 bits is rate 1/2 convolution encoded and clocked at 50 symbols per second. Each subframe of 292 bits after encoding results in 584 bits. For parameters and coding scheme, refer to below doc
\subsection{Interleaving}
Any burst errors during the data transmission can be corrected by interleaving. In matrix interleaving, input symbols are filled into a matrix column wise and read at the output row wise. This will spread the burst error, if any, during the transmission. For SPS, data is filled into matrix of size 73 by 8(73 columns, 8 rows).
\subsection{Sync word and Tail bits}
Each subframe has a 16 bit word synchronization pattern which is not encoded. Sync pattern is EB90 Hex. Tail bit consists of 6 zero value bits enabling completion of FEC decoding of each subframe in the receiver.

\section{Modulation}
\subsection{Standard Positioning Service}
The SPS signal is BPSK(1) modulated on L5 and S bands. The navigation data at data rate of 50 sps (1/2 rate FEC encoded) is modulo 2 added to PRN code chipped at 1.023 Mcps. The CDMA modulated code, modulates the L5 and S carriers at 1176.45 MHz and 2492.028 MHz respectively.
\subsection{Pseudo Random Noise codes(PRN)}
NavIC uses Gold codes fo SPS signal. They are generated using Linear Feedback Shift Registers. For L5 and S band, the code length is 1ms and consists of 1023 chips. The code is chipped at 1.023 Mcps. Two polynomials G1 and G2 are used to generate the gold code sequence. G2's initial state provides unique PRN code for each satellite. All bits of G1 are initialized as 1. G1 and G2 are XOR'ed to generate final 1023 chip long PRN sequence, the time period being 1ms. For more information refer to chapter 4 in the doc.
\subsection{Baseband Modulation}
The carrier signal is modulated by BPSK(1), Data channel BOC(5,2), and Pilot Channel BOC(5,2). To have a constant envelop when passed through power amplifier, we add additional signal called interplex signal.
\subsubsection{Mathematical Equations}
SPS Data Signal
\begin{equation}
	s_{sps}(t) = \sum_{i=-\infty}^{\infty} c_{sps}(\abs{i}_{L\_sps}) . d_{sps}(\sbrak{i}_{CD\_sps}) . rect_{T_{c,sps}}(t-iT_{c,sps})
\end{equation}
RS BOC Pilot Signal
\begin{equation}
	s_{rs\_p}(t) = \sum_{i=-\infty}^{\infty} c_{rs\_p}(\abs{i}_{L\_rs\_p}) . rect_{Tc,rs\_p}(t-iT_{c,rs\_p}). sc_{rs\_p}(t,0)
\end{equation}
RS BOC Signal
\begin{equation}
	s_{rs\_d}(t) = \sum_{i=-\infty}^{\infty} c_{rs\_d}(\abs{i}_{L\_rs\_d}) . d_{rs\_d}(\sbrak{i}_{CD\_rs\_d}). rect_{T_{c,rs\_d}}(t-iT_{c,rs\_d}). sc_{rs_d}(t,0)
\end{equation}
The sub-carrier is defined as:
\begin{equation}
	sc_x(t,\phi) = sgn\sbrak{sin(2 \pi f_{sc,x}t + \phi)}
\end{equation}

The IRNSS RS data and pilot BOC signals are sinBOC. Hence the subcarrier phase $\phi=0$.
The complex envelope of composite signal with Interplex signal (I(t)) is:
\begin{equation}
s(t) = \dfrac{1}{3} \sbrak{\sqrt{2} (s_{sps}(t) + s_{rs\_p}(t)) + j(2. s_{rs\_d}(t) – I(t))} 
\end{equation}

The Interplex signal $I(t)$ is generated to realize the constant envelope composite signal. The operation $\abs{i}_X$ gives the code chip index for any signal. Similarly $[i]_X$ gives data bit index for any signal.
Symbol definitions are given in below table \ref{table:symbdesc}.

\begin{table}[h]
%\centering
%%%%%%%%%%%%%%%%%%%%%%%%%%%%%%%%%%%%%%%%%%%%%%%%%%%%%%%%%%%%%%%%%%%%%%
%%                                                                  %%
%%  This is the header of a LaTeX2e file exported from Gnumeric.    %%
%%                                                                  %%
%%  This file can be compiled as it stands or included in another   %%
%%  LaTeX document. The table is based on the longtable package so  %%
%%  the longtable options (headers, footers...) can be set in the   %%
%%  preamble section below (see PRAMBLE).                           %%
%%                                                                  %%
%%  To include the file in another, the following two lines must be %%
%%  in the including file:                                          %%
%%        \def\inputGnumericTable{}                                 %%
%%  at the beginning of the file and:                               %%
%%        \input{name-of-this-file.tex}                             %%
%%  where the table is to be placed. Note also that the including   %%
%%  file must use the following packages for the table to be        %%
%%  rendered correctly:                                             %%
%%    \usepackage[latin1]{inputenc}                                 %%
%%    \usepackage{color}                                            %%
%%    \usepackage{array}                                            %%
%%    \usepackage{longtable}                                        %%
%    \usepackage{calc}                                             %%
%%    \usepackage{multirow}                                         %%
%%    \usepackage{hhline}                                           %%
%%    \usepackage{ifthen}                                           %%
%%  optionally (for landscape tables embedded in another document): %%
%%    \usepackage{lscape}                                           %%
%%                                                                  %%
%%%%%%%%%%%%%%%%%%%%%%%%%%%%%%%%%%%%%%%%%%%%%%%%%%%%%%%%%%%%%%%%%%%%%%



%%  This section checks if we are begin input into another file or  %%
%%  the file will be compiled alone. First use a macro taken from   %%
%%  the TeXbook ex 7.7 (suggestion of Han-Wen Nienhuys).            %%
\def\ifundefined#1{\expandafter\ifx\csname#1\endcsname\relax}


%%  Check for the \def token for inputed files. If it is not        %%
%%  defined, the file will be processed as a standalone and the     %%
%%  preamble will be used.                                          %%
\ifundefined{inputGnumericTable}

%%  We must be able to close or not the document at the end.        %%
	\def\gnumericTableEnd{\end{document}}


%%%%%%%%%%%%%%%%%%%%%%%%%%%%%%%%%%%%%%%%%%%%%%%%%%%%%%%%%%%%%%%%%%%%%%
%%                                                                  %%
%%  This is the PREAMBLE. Change these values to get the right      %%
%%  paper size and other niceties.                                  %%
%%                                                                  %%
%%%%%%%%%%%%%%%%%%%%%%%%%%%%%%%%%%%%%%%%%%%%%%%%%%%%%%%%%%%%%%%%%%%%%%

	\documentclass[12pt%
			  %,landscape%
                    ]{report}
       \usepackage[latin1]{inputenc}
       \usepackage{fullpage}
       \usepackage{color}
       \usepackage{array}
       \usepackage{longtable}
       \usepackage{calc}
       \usepackage{multirow}
       \usepackage{hhline}
       \usepackage{ifthen}

	\begin{document}


%%  End of the preamble for the standalone. The next section is for %%
%%  documents which are included into other LaTeX2e files.          %%
\else

%%  We are not a stand alone document. For a regular table, we will %%
%%  have no preamble and only define the closing to mean nothing.   %%
    \def\gnumericTableEnd{}

%%  If we want landscape mode in an embedded document, comment out  %%
%%  the line above and uncomment the two below. The table will      %%
%%  begin on a new page and run in landscape mode.                  %%
%       \def\gnumericTableEnd{\end{landscape}}
%       \begin{landscape}


%%  End of the else clause for this file being \input.              %%
\fi

%%%%%%%%%%%%%%%%%%%%%%%%%%%%%%%%%%%%%%%%%%%%%%%%%%%%%%%%%%%%%%%%%%%%%%
%%                                                                  %%
%%  The rest is the gnumeric table, except for the closing          %%
%%  statement. Changes below will alter the table's appearance.     %%
%%                                                                  %%
%%%%%%%%%%%%%%%%%%%%%%%%%%%%%%%%%%%%%%%%%%%%%%%%%%%%%%%%%%%%%%%%%%%%%%

\providecommand{\gnumericmathit}[1]{#1} 
%%  Uncomment the next line if you would like your numbers to be in %%
%%  italics if they are italizised in the gnumeric table.           %%
%\renewcommand{\gnumericmathit}[1]{\mathit{#1}}
\providecommand{\gnumericPB}[1]%
{\let\gnumericTemp=\\#1\let\\=\gnumericTemp\hspace{0pt}}
 \ifundefined{gnumericTableWidthDefined}
        \newlength{\gnumericTableWidth}
        \newlength{\gnumericTableWidthComplete}
        \newlength{\gnumericMultiRowLength}
        \global\def\gnumericTableWidthDefined{}
 \fi
%% The following setting protects this code from babel shorthands.  %%
 \ifthenelse{\isundefined{\languageshorthands}}{}{\languageshorthands{english}}
%%  The default table format retains the relative column widths of  %%
%%  gnumeric. They can easily be changed to c, r or l. In that case %%
%%  you may want to comment out the next line and uncomment the one %%
%%  thereafter                                                      %%
\providecommand\gnumbox{\makebox[0pt]}
%%\providecommand\gnumbox[1][]{\makebox}

%% to adjust positions in multirow situations                       %%
\setlength{\bigstrutjot}{\jot}
\setlength{\extrarowheight}{\doublerulesep}

%%  The \setlongtables command keeps column widths the same across  %%
%%  pages. Simply comment out next line for varying column widths.  %%
\setlongtables

\setlength\gnumericTableWidth{%
	80pt+%
	180pt+%
0pt}
\def\gumericNumCols{2}
\setlength\gnumericTableWidthComplete{\gnumericTableWidth+%
         \tabcolsep*\gumericNumCols*2+\arrayrulewidth*\gumericNumCols}
\ifthenelse{\lengthtest{\gnumericTableWidthComplete > \linewidth}}%
         {\def\gnumericScale{\ratio{\linewidth-%
                        \tabcolsep*\gumericNumCols*2-%
                        \arrayrulewidth*\gumericNumCols}%
{\gnumericTableWidth}}}%
{\def\gnumericScale{1}}

%%%%%%%%%%%%%%%%%%%%%%%%%%%%%%%%%%%%%%%%%%%%%%%%%%%%%%%%%%%%%%%%%%%%%%
%%                                                                  %%
%% The following are the widths of the various columns. We are      %%
%% defining them here because then they are easier to change.       %%
%% Depending on the cell formats we may use them more than once.    %%
%%                                                                  %%
%%%%%%%%%%%%%%%%%%%%%%%%%%%%%%%%%%%%%%%%%%%%%%%%%%%%%%%%%%%%%%%%%%%%%%

\ifthenelse{\isundefined{\gnumericColA}}{\newlength{\gnumericColA}}{}\settowidth{\gnumericColA}{\begin{tabular}{@{}p{80pt*\gnumericScale}@{}}x\end{tabular}}
\ifthenelse{\isundefined{\gnumericColB}}{\newlength{\gnumericColB}}{}\settowidth{\gnumericColB}{\begin{tabular}{@{}p{180pt*\gnumericScale}@{}}x\end{tabular}}

\begin{longtable}[c]{%
	b{\gnumericColA}%
	b{\gnumericColB}%
	}

%%%%%%%%%%%%%%%%%%%%%%%%%%%%%%%%%%%%%%%%%%%%%%%%%%%%%%%%%%%%%%%%%%%%%%
%%  The longtable options. (Caption, headers... see Goosens, p.124) %%
%	\caption{The Table Caption.}             \\	%
% \hline	% Across the top of the table.
%%  The rest of these options are table rows which are placed on    %%
%%  the first, last or every page. Use \multicolumn if you want.    %%

%%  Header for the first page.                                      %%
%	\multicolumn{2}{c}{The First Header} \\ \hline 
%	\multicolumn{1}{c}{colTag}	%Column 1
%	&\multicolumn{1}{c}{colTag}	\\ \hline %Last column
%	\endfirsthead

%%  The running header definition.                                  %%
%	\hline
%	\multicolumn{2}{l}{\ldots\small\slshape continued} \\ \hline
%	\multicolumn{1}{c}{colTag}	%Column 1
%	&\multicolumn{1}{c}{colTag}	\\ \hline %Last column
%	\endhead

%%  The running footer definition.                                  %%
%	\hline
%	\multicolumn{2}{r}{\small\slshape continued\ldots} \\
%	\endfoot

%%  The ending footer definition.                                   %%
%	\multicolumn{2}{c}{That's all folks} \\ \hline 
%	\endlastfoot
%%%%%%%%%%%%%%%%%%%%%%%%%%%%%%%%%%%%%%%%%%%%%%%%%%%%%%%%%%%%%%%%%%%%%%

\hhline{|-|-}
	 \multicolumn{1}{|p{\gnumericColA}|}%
	{\gnumericPB{\raggedright}\gnumbox[l]{\hspace*{1cm}\textbf{Symbol}}}
	&\multicolumn{1}{p{\gnumericColB}|}%
	{\gnumericPB{\raggedright}\gnumbox[l]{\hspace*{2cm}\textbf{Definition}}}
\\
\hhline{|--|}
	 \multicolumn{1}{|p{\gnumericColA}|}%
	{\gnumericPB{\raggedright}\gnumbox[l]{\hspace{1cm}A}}
	&\multicolumn{1}{p{\gnumericColB}|}%
	{\gnumericPB{\raggedright}\gnumbox[l]{ received signal amplitude}}
\\
\hhline{|--|}
	 \multicolumn{1}{|p{\gnumericColA}|}%
	{\gnumericPB{\raggedright}\gnumbox[l]{\hspace{1cm}$f_c$}}
	&\multicolumn{1}{p{\gnumericColB}|}%
	{\gnumericPB{\raggedright}\gnumbox[l]{carrier frequency}}
\\
\hhline{|--|}
	 \multicolumn{1}{|p{\gnumericColA}|}%
	{\gnumericPB{\raggedright}\gnumbox[l]{\hspace{1cm}$f_{sub}$}}
	&\multicolumn{1}{p{\gnumericColB}|}%
	{\gnumericPB{\raggedright}\gnumbox[l]{subcarrier frequency}}
\\
\hhline{|--|}
	 \multicolumn{1}{|p{\gnumericColA}|}%
	{\gnumericPB{\raggedright}\gnumbox[l]{\hspace{1cm}t}}
	&\multicolumn{1}{p{\gnumericColB}|}%
	{\gnumericPB{\raggedright}\gnumbox[l]{time}}
\\
\hhline{|--|}
	 \multicolumn{1}{|p{\gnumericColA}|}%
	{\gnumericPB{\raggedright}\gnumbox[l]{\hspace{1cm}q}}
	&\multicolumn{1}{p{\gnumericColB}|}%
	{\gnumericPB{\raggedright}\gnumbox[l]{ phase offset}}
\\
\hhline{|--|}
	 \multicolumn{1}{|p{\gnumericColA}|}%
	{\gnumericPB{\raggedright}\gnumbox[l]{\hspace{1cm}s(t)}}
	&\multicolumn{1}{p{\gnumericColB}|}%
	{\gnumericPB{\raggedright}\gnumbox[l]{BPSK signal transmitted data(-1,1)}}
\\
\hhline{|-|-|}
\end{longtable}

\ifthenelse{\isundefined{\languageshorthands}}{}{\languageshorthands{\languagename}}
\gnumericTableEnd

\caption{Symbol Description}
\label{table:symbdesc}
\end{table}

The functions for data generation and baseband modulation are present in the below code,
\begin{lstlisting}
    codes/transmitter/transmitter.py
\end{lstlisting}
